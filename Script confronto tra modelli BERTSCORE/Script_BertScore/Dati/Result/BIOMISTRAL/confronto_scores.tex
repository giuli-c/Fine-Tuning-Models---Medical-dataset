
\renewcommand{\arraystretch}{1.7}  % Aumenta lo spazio tra le righe (1.7 è il fattore di scala)

\centering
\begin{tabularx}{\textwidth}{|>{\centering\columncolor{blue!20}}p{1.7cm}|>{\centering\arraybackslash}X|>{\centering\arraybackslash}X|>{\centering\arraybackslash}X|}   % fisso la dimensione della colonna 1 e centro il nome in tutte
\hline
\rowcolor{blue!30}  % Colora la prima riga (intestazione)
\begin{tabular}{|c|c|c|c|}
\hline
\fontsize{13}{14}\textbf{Epoche} & \fontsize{13}{14}\textbf{Mean Precision} & \fontsize{13}{14}\textbf{Mean Recall} & \fontsize{13}{14}\textbf{Mean F1} \\ \fontsize{12}{12}\selectfont
\hline
Base & 0.556259 & 0.695039 & 0.617517 \\ \fontsize{12}{12}\selectfont
1 & 0.798099 & 0.833858 & 0.815587 \\ \fontsize{12}{12}\selectfont
2 & 0.807400 & 0.781000 & 0.793400 \\ \fontsize{12}{12}\selectfont
3 & 0.808800 & 0.744000 & 0.774600 \\ \fontsize{12}{12}\selectfont
4 & 0.818000 & 0.795400 & 0.806000 \\ \fontsize{12}{12}\selectfont
5 & 0.815800 & 0.793900 & 0.804100 \\ \fontsize{12}{12}\selectfont
6 & 0.875142 & 0.871439 & 0.873286 \\ \fontsize{12}{12}\selectfont
\hline
\end{tabular}

\end{tabularx}
